\documentclass[french, 12pt, titlepage]{article}
\usepackage[utf8]{inputenc}
\usepackage[T1]{fontenc}
\usepackage[francais]{babel}
\usepackage{enumitem}
\usepackage{color}
\usepackage[colorlinks,linkcolor=grey, urlcolor=gray, citecolor=red, breaklinks, pagebackref]{hyperref}
\usepackage{graphicx}
\usepackage{listings} %for code citations
\usepackage{fancyhdr} %for footers / headers
\usepackage{amsfonts}
\usepackage{amsmath}
\usepackage{amsthm}
\usepackage{amssymb}
\usepackage{listings}
\usepackage{verbatim}
\usepackage{array}
\usepackage{tabularx}
\usepackage{mathabx}
\usepackage[french,ruled]{algorithm2e}
\usepackage{pdfpages}
\newenvironment{itemH}[0]{\begin{itemize}[label=$\bullet$, font=\color{black} \large]}{\end{itemize}}


\definecolor{gray}{rgb}{0.8,0.8,0.8}
\definecolor{grey}{rgb}{0.3,0.5,0.5}

\lstset{language=C, frame=shadowbox, basicstyle=\small, backgroundcolor=\color{white}, rulesepcolor=\color{gray}, breaklines=true, morecomment=[l]{//}, numbers=left, numberstyle=\tiny}

\SetKwInput{KwIn}{Entrées}
\SetKwInput{KwOut}{Sorties}
\SetKw{Return}{Retourner}
\SetKwRepeat{Repeat}{Répéter}{jusqu'à}
\DeclareMathOperator{\pgcd}{pgcd}
\DeclareMathOperator{\ppcm}{ppcm}
\DeclareMathOperator{\Ima}{Im}
\newcommand{\R}{\mathbb{R}}
\newcommand{\Z}{\mathbb{Z}}
\newcommand{\N}{\mathbb{N}}
\newtheorem{definition}{D{\'e}finition}
\newtheorem{theoreme}{Th{\'e}or{\`e}me}
\newtheorem{lemme}{Lemme}
\newtheorem{ep}{Esquisse de preuve}
\newtheorem{notation}{Notation}
\newtheorem{example}{Exemple}


\begin{document}
\title{Factorisation des entiers}
\author{Vincent Dalsheimer \and
	Gaëtan Pradel}
\date{Année 2016 - Semestre 10}
\maketitle
\tableofcontents

\newpage
\section{Introduction}


Factoriser un entier de manière efficace est un vieux problème d'arithmétique. Nombre de célèbres mathématiciens s'y sont confrontés
({\'E}ratosthène, Fermat, Gauss, Mersenne, ...) mais leur intérêt était purement mathématique.

Toutefois, avec l'apparition de la cryptologie, le
problème prend une toute autre ampleur. En effet, certains algorithmes
de chiffrement reposent sur une clef privée qui est la factorisation de
la clef publique $n$ (RSA par exemple). Réussir à factoriser $n$ implique donc de pouvoir casser ces algorithmes.

\begin{theoreme}[Théorème fondamental de l'arithmétique]
  Tout entier strictement positif peut être écrit comme un produit de nombres premiers d'une unique façon, à l'ordre près des facteurs.
\end{theoreme}

Ce théorème implique donc que tout entier $n$ peut être factorisé de manière unique, justifiant alors la recherche d'algorithmes de factorisation.
La performance de ces algorithmes est cruciale en cryptologie. En effet, c'est grâce à celle-ci que l'on peut déterminer la taille minimum nécessaire des clefs des algorithmes de chiffrement afin qu'ils ne soient pas
cassables en un temps raisonnable.

Le plus ancien de ces algorithmes est la méthode de factorisation dite naïve. C'est la méthode de factorisation la plus intuitive. Depuis, par souci d'efficacité, de nombreux
algorithmes plus rapides ont été trouvés, comme l'algorithme $p - 1$ de Pollard, le crible de Dixon ou encore le crible quadratique.

Nous étudierons par la suite le chiffrement RSA, ainsi que les quatre algorithmes sus-mentionnés.


\section{Un algorithme de chiffrement utilisant une clef publique $n = pq$ : RSA}

Le chiffrement RSA est un chiffrement asymétrique, c'est-à-dire qu'il utilise une paire de clefs, qui sont des nombres entiers, composée d'une clef publique pour chiffrer et d'une clef privée pour déchiffrer.
La clef privée peut être aussi utilisée pour signer un message. La clef publique est de la forme $n = pq,$ et la clef privée peut être trouvée à partir de $p$ et $q.$ Ainsi, factoriser $n$ revient à casser RSA.

\subsection{Création des clefs}

\begin{itemH}
\item Choisir $p$ et $q,$ deux nombres premiers entiers distincts ;
\item calculer leur produit $n = pq$
\item calculer $\phi (n) = (p - 1)(q - 1)$ qui est la valeur de l'indicatrice d'Euler en $n$ ;
\item choisir un entier naturel $e$ premier avec $\phi (n)$ et strictement inférieur à $\phi (n)$ ;
\item calculer l'entier naturel $d,$ inverse de $e$ modulo $ \phi (n),$ et strictement inférieur à $\phi (n)$ ; $d$ peut se calculer efficacement par l'algorithme d'Euclide étendu.
\end{itemH}

Le couple $(n, e)$ est la clef publique et $d$ est la clef privée.

La sécurité de cet algorithme repose sur le fait que la factorisation (de grands nombres) est un problème difficile, c'est-à-dire qu'on ne peut pas le résoudre en temps polynomial.

\subsection{Chiffrement}
Soit $m$ un message à chiffrer. Le message chiffré $c$ de $m$ sera :
\[ c \equiv m^e \pmod n .\]
\subsection{Déchiffrement}

Soit le message chiffré $c$ comme ci-dessus. Pour retrouver le message $m$ on fait:
\[ c^d \equiv m^{ed} \equiv m \pmod n.\]

\subsection{Justification}

La démonstration repose sur le petit théorème de Fermat, à savoir que comme $p$ et $q$ sont deux nombres premiers, si $m$ n'est pas un multiple de $p$ on a la première égalité ci-dessous, et la seconde s'il n'est pas un multiple de $q$ :
\[ m^{p-1} \equiv 1 \pmod p\ ,\ \ m^{q-1} \equiv 1 \pmod q.\]
En effet
\[ c^d \equiv (m^e)^d \equiv m^{ed} \pmod n.\]
Or
\[ed \equiv 1 \pmod{(p-1)(q-1)}\]
ce qui signifie que pour un entier $k$
\[ed = 1 + k(p-1)(q-1),\]
donc, si $m$ n'est pas multiple de $p$ d'après le petit théorème de Fermat
\[m^{ed} \equiv m^{1+k(p-1)(q-1) }\equiv m\cdot \left(m^{p-1}\right)^{k(q-1)}\equiv m \pmod p\]
et de même, si $m$ n'est pas multiple de $q$
\[m^{ed}\equiv m \pmod q.\]

\section{Méthode de factorisation naïve}

Pour factoriser un entier $n,$ la méthode la plus naïve et la plus naturelle consiste à faire les divisions euclidiennes successives de $n$ par les entiers $i,$ $2 \leq i \leq \lfloor\sqrt{n}\rfloor.$ Si le reste est nul, on garde $i$ de côté et on recommence avec $\frac{n}{i}.$ Sinon, on passe à $i + 1.$ Quand $\frac{n}{i} = 1,$ la factorisation est terminée et les facteurs de $n$ sont donc les $i$ qu'on a mis de côté.

Il est toutefois possible d'améliorer légèrement cet algorithme à l'aide du crible d'{\'E}ratosthène. Ce crible permet d'obtenir tous les nombres premiers inférieurs à un entier $n.$
On prend la liste de tous les entiers compris entre $2$ et $n,$ puis on garde $2$ mais on supprime tous ses multiples. On recommence la même opération avec l'entier le plus proche à ne pas avoir été déjà supprimé.
On continue ainsi jusqu'à $\lfloor\sqrt{n}\rfloor,$ entier à partir duquel tous les nombres restants dans la liste sont des premiers.

Une fois la liste des nombres premiers inférieurs à $n$ générée, on applique la même méthode de factorisation, mais en divisant $n$ non plus par les entiers $i,$ $2 \leq i \leq \lfloor\sqrt{n}\rfloor,$
mais par les premiers obtenus grâce au crible d'{\'E}ratosthène.

Pour des petits entiers (moins de dix chiffres), la méthode naïve reste la méthode de factorisation la plus efficace. Pour des facteurs plus grands, cet algorithme est généralement moins efficace
que ceux que nous allons étudier en suivant.

\section{L'algorithme $p-1$ de Pollard}

L'algorithme $p-1$ de Pollard est un algorithme de décomposition en produit de facteurs premiers. Cette méthode fonctionne seulement avec des nombres qui ont une forme particulière. Il trouve les facteurs $p$ dont $p-1$ est ultrafriable.

\begin{definition}[Entier friable]
Un entier strictement positif $n$ est dit $B$-friable ou $B$-lisse si tous ses facteurs premiers sont inférieurs ou égaux à $B.$ 
\end{definition}
\begin{example}
$90 = 2 \times 3^2 \times 5$ est $5$-friable car aucun de ses facteurs premiers ne dépasse 5.
Cette définition inclut les nombres qui ne figurent pas parmi les facteurs premier: par exemple, 12 est $5$-friable.
\end{example}
\begin{definition}[Entier ultrafriable]
Un nombre $n$ est dit $B$-superlisse ou $B$-ultrafriable si toute puissance $p^r$ d'un nombre premier qui divise $n$ vérifie: \[ p^r \leq B .\]
\end{definition}
\begin{example}
$720 = 2^4 \times 3^2 \times 5$ est $5$-friable mais pas $5$-ultrafriable ($3^2 = 9 > 5$). Par contre il est $16$-ultrafriable puisque sa plus grande puissance de facteur premier est $2^4 = 16.$
\end{example}

\begin{lemme}\label{leme}
Si $m$ est $B$-ultrafriable pour un certain seuil $B,$ alors $m | \ppcm(1,...,B).$
\end{lemme}

\begin{lemme}[Lemme de Gauss]
Si un nombre entier $a$ divise le produit de deux autres nombres entiers $b$ et $c,$ et si $a$ est premier avec $b,$ alors $a$ divise $c.$
\end{lemme}

\subsection{Principe de l'algorithme}

\begin{theoreme}[Petit Théorème de Fermat]
Soient $p$ un nombre premier et $a$ est un entier non divisible par $p.$ Alors $a^{p-1} - 1$ est un multiple de $p.$ C'est-à-dire : \[a^{p-1} = 1 \pmod p .\]
\end{theoreme}
Par le petit théorème de Fermat, nous savons que \[a^{p-1} = 1 \pmod p \] pour $a$ premier avec $p.$ 

Cela implique que pour tout multiple $M$ de $p-1$ on a: \[ a^M - 1 \equiv 0 \pmod p \text{ car } a^{k(p-1)} - 1 = (a^{p-1} - 1 )\sum\limits_{i=0}^{k-1} a^{i(p-1)} . \]

Soit $n$ un entier divisible par un nombre premier $p,$ avec $n \neq p$
et $p-1 \text{ } B\text{-ultrafriable}.$
D'après le lemme \ref{leme}, si l'on pose $M = ppcm(1, ..., B)$, on a: \[ a^M \equiv 1 \pmod p \text{ pour tout } a \text{ premier avec } p . \]

Ainsi, $p$ divise $a^M - 1$ et donc le pgcd de $n$ et $a^M - 1$ est
divisible par $p.$
En revanche, il est possible que le pgcd soit égal à $n$ lui-même auquel cas, on n'obtient pas de facteur non trivial.

\subsection{Pseudo-code}

Choisir un résidu $x \pmod n$ au hasard et initialier $f$ à $1.$ $k$ fera office de compteur.
Définir une suite en posant $x_1 = x,$ $x_2 = x_1^2 \pmod n,$ $x_3 = x_2^3 \pmod n,$ ... Ainsi $x_ {k+1}$ est obtenu en élevant $x_k$ à la puissance $k+1$ modulo $n.$ Autrement dit $x_k = x^{k!}.$
Choisir une limite pour finir l'algorithme s'il ne trouve pas de résultats après un certain nombre d'essais.

\begin{algorithm}
\caption{Factorisation de $n$ par $p-1$ de Pollard}
\KwIn{Un entier $n$}
\KwOut{Un facteur $f$ de $n$}
\BlankLine
$x \gets Random([\![1,n-1]\!])$\;
$f = 1$\;
$k = 2$\;
$x \gets x^k$\;
$lim$\;
\Repeat{$p \neq 1$ || $k == lim$}{$x \gets x^{k!} \pmod n$\;$f = pgcd(x - 1, n)$\;$k = k + 1$\;}
\Return{$f$}
\end{algorithm}

\subsection{Exemple}

Nous factorisons le nombre $172189$ avec notre algorithme $p-1$ de Pollard.
On a $172189 = 409 \times 421$ et \[409 - 1 = 408 = 2^3 \times 3 \times 17 \] puis \[421-1 = 420 = 2^2 \times 3 \times 5 \times 7.\] 
Voici ce que l'on obtient avec cet exemple:

\begin{center}
\begin{tabular}{|c||c|c|c|c|c|c|c|}
\hline
$k$ & 1 & 2 & 3 & 4 & 5 & 6 & 7 \\
\hline
$x_k = x^{k!} \pmod n$ & 2 & 4 & 64 & 74883 & 27019 & 147176 & 45890 \\
\hline
$pgcd(x_k -1, n)$ & 1 & 1 & 1 & 1 & 1 & 1 & 421\\
\hline
\end{tabular}
\end{center}

Au premier tour on trouve donc $421.$ On le fait ensuite sur $172189 \div 421 = 409.$ Or avec un test de primalité, on voit directement que $421$ et $409$ sont premiers. Mais voici tout de même ce que l'on a en résultat avec $409:$ 

\begin{center}
\begin{tabular}{|c||c|c|c|c|c|c|c|}
\hline
$k$ & 1 & 2 & 3 & 4 & 5 & 6 & ... \\
\hline
$x_k = x^{k!} \pmod n$ & 2 & 4 & 64 & 36 & 25 & 345 & ... \\
\hline
$pgcd(x_k -1, n)$ & 1 & 1 & 1 & 1 & 1 & 1 & ...\\
\hline
\end{tabular}
\end{center}

L'algorithme ne trouve pas de facteur, c'est normal car $409$ est premier.

\subsection{Limites de l'algorithme}

Dans certains cas, l'algorithme nous renvoie le même nombre mis en entrée ou une factorisation incomplète de celui-ci, en effet, cela correspond aux cas où les $p-1$ ne sont pas ultrafriables.

Par exemple, avec le nombre $7345461,$ l'algorithme nous renvoie une factorisation incomplète.
La factorisation naïve nous renvoie $7345461 = 3 \times 563 \times 4349$ tandis que $p-1$ de Pollard nous renvoie $7345461 = 3 \times 2448487.$
A priori, on suppose donc que $562$ et $4348$ ne sont pas ultrafriables, et en effet:
$562 = 2 \times 281$ et $4348 = 2 \times 1087.$

\subsection{Large prime variation}
\subsubsection{Principe}
Il existe une variante de la méthode $p - 1$ de Pollard, appelée \textit{large prime variation}. 
Elle aboutit si $p - 1$ est $(B, C)-$friable.

\begin{definition}
  Soit $n, B, C$ des nombres entiers. On dit que n est $(B, C)-$friable s'il est le produit de premiers $\leq B$ et d'au plus un premier dans l'intervalle $]B, C]$.
\end{definition}
On démarre l'algorithme comme précédemment. On calcule $x_{k + 1} = x^{k + 1} \mod n$ jusqu'à $x_B.$ Ensuite on calcule $y_1 = x_B^{l_1} \mod n$, $y_2 = x_B^{l_2},$ ...; où les $l_i$ sont les premiers dans $]B,C]$ par ordre croissant.
L'avantage de cette méthode est que pour calculer $y_2$ on n'a pas besoin d'une exponentiation modulaire. En effet \[ y_2 = x_B^{l_2} = x_B^{l_1 + (l_2 - l_1)} = y_1 x_B^{l_2 - l_1} \mod n.\]
Or, la différence entre deux nombres premiers consécutifs est très petite. On calcule donc une fois pour toutes les premières puissances de $x_B$ soit $z_j = x_B^j$ et on a \[ y_{k + 1} = y_k z_{l_{k+1} - l_k} \mod n \] qui se calcule au prix d'une seule multiplication.
\subsubsection{Exemple}

Soit $n = 2721749 = pq$ avec $p = 2671$ et $q = 1019$. On a: \[p - 1 = 2 \times 3 \times 5 \times 89\] est $(B, C)-$friable avec $B = 7$ et $C = 100$ par exemple.

On calcule
\begin{align*}
  x_1 &= 2 \\
  x_2 &= x_1^2 = 4\\
  x_3 &= 64\\
  x_4 &= 446722\\ 
  x_5 &= 1416863\\
  x_6 &= 715291 \\
  x_7 &= 795854 \\
\end{align*}

On pose $y = x_7$.

On élève $y$ à la puissance $l_i$ pour tous les premiers $l_i$ entre $11$ et $97$.

On voit sans peine que la différence entre deux tels premiers consécutifs est au plus de $8.$

On calcule donc les huit premières puissances de $y$ soit:

\begin{align*}
  z_1 &= y = 795854\\
  z_2 &= y^2 = 2657777\\
  z_3 &= y^3 = 664706\\
  z_4 &= 1628037\\
  z_5 &= 2034144\\
  z_6 &= 1664270\\
  z_7 &= 1281471\\
  z_8 &= 2696942\\
\end{align*}

On peut maintenant calculer
\begin{align*}
  y_1 &= y^{11} = 1715449\\
  y_2 &= y^{13} = y_1 z_2 = 216252\\
  y_3 &= y^{17} = y_2 z_4 = 2580676\\
  & ... \\
  y_{20} &= y^{89} = y_{19} z_6 = 1279410\\
\end{align*}
On trouve que $ \pgcd (y_{20} - 1, n) = 2671.$





\section{Crible de Dixon}

Le crible de Dixon se base sur la recherche de congruences de
carrés. Son fonctionnement s'inspire de celui de l'algorithme de
factorisation de Fermat qui consistait à écrire $n$ comme la
différence de deux carrés. On avait alors : \[n = a^2 - b^2 = (a - b)(a + b).\]

\subsection{Principe de l'algorithme}

On cherche deux entiers $a$ et $b$ tels que $a^2 \equiv b^2 \pmod n$ et $a \nequiv b \pmod n.$ Un facteur de $n$ pourra alors être trouvé en calculant
$\pgcd(a - b, n).$

Pour cela, on choisit une borne $B$ et on note $k$ le nombre de
premiers $p$ inférieurs à $B.$ On appelle $P$ l'ensemble de ces
premiers (de cardinal $k,$ donc).

On prend ensuite un $x$ aléatoirement dans $[1, n - 1]$ et
on calcule $y \equiv x^2 \pmod n.$ Si $y$ est $B$-friable, on garde le
couple $(x, y)$ appelé \textit{relation}. Appelons $R$ l'ensemble des
relations et $m$ son cardinal. On recommence l'opération jusqu'à
avoir $m > k.$ On définit $v_{p, i}$ par $x_i^2 \pmod n = y_i = \prod\limits_{p \in P} p^{v_{p, i}}.$

On construit ensuite la matrice $M = (v_{p, i} \pmod 2)_{p \in P, 1
  \leq i \leq m},$ puis on trouve un vecteur non nul $(e_1, ..., e_m)^t$
dans le noyau de $M.$
On a alors : \[\prod\limits_{i=1}^m x_i^{2e_i} = \prod\limits_{i=1}^m \prod\limits_{p \in P} p^{v_{p, i}}e_i = \prod\limits_{p \in P} p^{\sum\limits_{i=1}^m v_{p, i}e_i} \pmod n.\]

Or $(e_1, ..., e_m)$ annule $M,$ donc : \[\sum\limits_{i=1}^m v_{p,
  i}e_i = 0 \pmod 2.\]

On a donc une congruence de carrés : \[a^2 = b^2 \pmod n \text{ avec } a = \prod\limits_{i=1}^m x_i^{e_i} \text{ et } b = \prod\limits_{p \in P} p^{\frac{1}{2} \sum\limits_{i=1}^m v_{p, i}e_i}\] qui nous donnera un facteur
de $n$ en calculant $\pgcd(a - b, n).$

\subsection{Exemples}

\subsubsection{Exemple simple}

Factorisons $n = 7081$ avec $B = 3.$ Considérons $-1$ comme un
premier. Les entiers $B$-friables sont donc de la forme $\pm2^k3^l,$
$k, l \in \N.$ 

On trouve trois carrés modulo $n$ qui sont 3-friables :
\begin{alignat*}{2}
4486^2 & \equiv && -\!2 \times 3 \pmod n, \\
1857^2 & \equiv && \text{ } 2 \pmod n, \\
2645^2 & \equiv && -\!3 \pmod n.
\end{alignat*}

On construit alors \[M = \begin{tabular}{c|ccc}
                            & 4486 & 1857 & 2645 \\
                           \hline
                           -1 & 1 & 0 & 1 \\
                           2 & 1 & 1 & 0 \\
                           3 & 1 & 0 & 1 \\
                         \end{tabular}\]

Le vecteur colonne $(1, 1, 1)^t$ appartient au noyau de $M,$ on a donc
: \[(4486 \times 1857 \times 2645)^2 \equiv (-2 \times 3)^2 \pmod
n.\]

On calcule ensuite $\pgcd(4486 \times 1857 \times 2645 - (-2 \times
3), 7081).$ On trouve 73 qui est donc un facteur non-trivial de 7081.

\subsubsection{Exemple plus complexe effectué avec notre code}

On cherche à factoriser $n = 386170680385.$ On prend donc $B = 176.$

Avec notre code (cf. Annexe), on obtient une matrice $M$ de taille (41, 51) et les relations correspondantes. On calcule un vecteur non nul de $\ker(M),$ et on obtient alors quatorze relations qui nous intéressent :
\begin{alignat*}{2}
35158967711 & \leftrightarrow (3, 11, 13, 83, 97), \\
235899315194 & \leftrightarrow (2, 13, 19, 37, 67, 109), \\
304776732510 & \leftrightarrow (3, 5, 17, 47, 109), \\
107665685121 & \leftrightarrow (2, 31, 59, 127), \\
65236108294 & \leftrightarrow (7, 53, 97, 107, 139), \\
81957238850 & \leftrightarrow (2, 5, 17, 19, 31, 67), \\
40726143234 & \leftrightarrow (3, 7, 97, 109, 131, 163), \\
243441301943 & \leftrightarrow (2, 7, 23, 47, 109), \\
259898178015 & \leftrightarrow (2, 7, 17, 107, 167), \\
378849397508 & \leftrightarrow (2, 3, 37, 83, 127, 163), \\
107786280772 & \leftrightarrow (19, 29, 59), \\
91195582280 & \leftrightarrow (2, 5, 19, 23, 131, 139), \\
375263838281 & \leftrightarrow (11, 17, 23, 29, 53, 167), \\
180950449505 & \leftrightarrow (2, 5, 23, 97). \\
\end{alignat*}

On multiplie ensuite tous les membres de gauche et on leur soustrait le produit des membres de droite élevés à la puissance $\frac{\text{nombre d'apparitions}}{2}.$
Par exemple, $3$ qui apparaît dans quatre relations sera élevé au carré.

On calcule ensuite le pgcd du nombre trouvé (de l'ordre de $10^{156}$) et de $n,$ et on obtient $1645.$ 
Or, $\frac{n}{1645} = 234754213$ et on a donc une factorisation de $n = 386170680385.$
Pour retrouver sa décomposition en facteurs premiers, il suffit de décomposer de la même façon les deux facteurs trouvés, à savoir $1645$ et $234754213.$
On trouve alors $1645 = 5 \times 7 \times 47$ et $234754213 = 10007 \times 23459.$

On a donc la décomposition en facteurs premiers de $n$ :
\[386170680385 = 5 \times 7 \times 47 \times 10007 \times 23459.\]

\subsection{Pseudo-code}

\begin{algorithm}
\caption{Factorisation de $n$ par le crible de Dixon}
\KwIn{Un entier $n$ et une borne $B$}
\KwOut{Un facteur de $n,$ ou 1 ou $n$}
\BlankLine
$P = \{p \text{ premiers } | p < B\}$\;
$k = \#P$\;
$L = liste\_relations$\;
\For{$i \in \{0, 1, ...,  k + 10\}$}{$x \gets Random([\![1,n-1]\!])$\; $y \equiv x^2 \pmod n$\;
\If{$y \text{ }B \text{-friable }$}{$L[i] \gets  x$}}
$M = (v_{p, i} \pmod 2)_{p \in P, 1 \leq i \leq k + 10} \text{ où } L[i]^2 \pmod n = \prod\limits_{p \in P}p^{v_{p,i}}$\;
Calculer le noyau de M. Soit $e = (e_1, ..., e_{k + 10}) \in \ker(M),$ c'est-à-dire :\\
$(e_1, ..., e_{k + 10}) \text{ tel que } M(e_1, ..., e_{k + 10})^t \equiv 0 \pmod 2$\;
$u = \prod\limits_{i = 1}^m L[i]^{e_i}$\;
$v = \prod\limits_{p \in P} p^{\frac{1}{2} \sum\limits_{i=1}^m v_{p, i}e_i}$\;
\Return{$\pgcd(u - v, n)$}
\end{algorithm}

\subsection{Précisions sur l'algorithme}

\subsubsection{Facteur non trivial}

L'algorithme présenté retourne un facteur de $n,$ éventuellement 1 ou
$n.$ Toutefois, si $n$ est impair et le produit d'au moins deux
premiers, le facteur trouvé est différent de 1 et $n$ avec une
probabilité supérieure à 1/2. 

En effet, supposons $n = pq$ avec $p$ et $q$ premiers. Supposons $a^2 \equiv b^2 \pmod n$
où $a \in [\![1, n-1]\!].$ On a donc le tableau de probabilités
suivant : 
\begin{center}
\begin{tabular}{l|c|c}
 & $a \equiv b \pmod p$ & $a \equiv -b \pmod p$ \\
\hline
$a \equiv b \pmod q$ & 1/4 & 1/4 \\
\hline
$a \equiv -b \pmod q$ & 1/4 & 1/4 \\
\end{tabular}
\end{center}
On obtiendra donc un facteur non trivial de $n$ dans la moitié des cas
(lorsque $a \equiv b \pmod p$ et $a \equiv -b \pmod q,$ ou $a \equiv -b \pmod p$ et $a \equiv b \pmod q$).

En fait, si $n$ a $m,$ $m \geq 2,$ facteurs premiers, la probabilité d'obtenir un
facteur non trivial est égale à $1 - \frac{1}{m^2} \geq \frac{1}{2}$
($1 - \textit{probabilité que } a \equiv b \pmod {p_1, p_2, ..., p_m} \textit{ ou que } a \equiv -b \pmod {p_1, p_2, ..., p_m}).$

Dans les faits, on teste d'abord si $n$ est une puissance exacte en calculant numériquement les racines de $n$ ($\sqrt{n},$ $\sqrt[3]{n},$ etc), puis s'il est premier avec un algorithme de primalité (Rabin-Miller, par
exemple), et s'il n'est ni une puissance exacte, ni (vraisemblablement) pas premier, on peut alors le
factoriser avec le crible de Dixon.


\subsubsection{Choix de $B$}

Le choix de $B$ est crucial. En effet, si $B$ est trop petit, on
n'aura pas suffisamment de relations différentes pour trouver un
vecteur non nul dans le noyau de $M,$ et plus $B$ augmente, plus
l'algorithme mettra du temps à trouver des congruences entre carrés. 

\begin{lemme}
La probabilité qu'un entier $n$ inférieur à un entier $C$ soit
$B$-friable est à peu près égale à $u^{-u},$ où $u = \frac{\log(C)}{\log(B)}.$
\end{lemme}

\begin{proof}[Esquisse de preuve]

Soit $\Psi(C, B) = \#\{x \in [\![1, C]\!]; x \text{ est } B\text{-friable}\}.$
La probabilité $P$ qu'un entier $n$ inférieur à $C$ soit $B$-friable est donc égale à $\frac{\Psi(C, B)}{C}.$

Calculons maintenant $P$ de manière approximative. On sait que $B^u =
C$ et on définit donc 
\begin{alignat*}{3}
\Pi : && [\![1, B]\!]^u & \to & [\![1, C]\!] \\
&& (x_1, x_2, ..., x_u) & \mapsto & \prod\limits_{i = 1}^m x_i
\end{alignat*}

L'image de $\Pi$ est donc l'ensemble des entiers $B$-friables inférieurs à $C.$ On a donc $\#(\Ima \Pi) = \Psi(C, B).$ Calculons alors $\#(\Ima \Pi)$ de manière approximative :
\[\#(\Ima \Pi) = \frac{\#([\![1, B]\!]^u)}{\#\{\text{\!éléments qui ont la même image}\}}\]

Or, $\#([\![1, B]\!]^u) = B^u$ et
$\#\{\text{\!éléments qui ont la même image}\} \approx u!$ en admettant que le nombre d'éléments de $[\![1, B]\!]^u$ qui ont la même image est proche du nombre de permutations possibles dans cet ensemble.
Ainsi, $\#(\Ima \Pi) = \frac{B^u}{u!} = \frac{C}{u!}.$ On a donc :
\[P = \frac{\Psi(C, B)}{C} = \frac{\#(\Ima \Pi)}{C} \approx \frac{\frac{C}{u!}}{C} = \frac{1}{u!}\]

\begin{lemme}[Formule de Stirling]
$n! \approx \sqrt{2\pi n}\text{ }(\frac{n}{e})^n$ pour tout $n \in \N.$
\end{lemme}

La formule de Stirling peut aussi s'écrire $n! = n^n$ pour un grand entier $n$ car $\frac{\sqrt{2\pi n}}{e^n}$ est alors négligeable devant $n^n.$
On a donc $P \approx \frac{1}{u^u} \approx u^{-u}.$ 
\end{proof}

On peut maintenant déterminer la valeur la plus adéquate de $B$ pour le crible.

\begin{lemme}
La valeur optimale de $B$ dans le crible de Dixon est $\text{e}^{\sqrt{\log(n)}}.$
\end{lemme}

\begin{proof}

Le temps $T$ que met l'algorithme est à peu près égal à
\[\textit{nombre de relations } \times \frac{1}{\textit{probabilité que
  } Y \textit{soit lisse }}.\] Le nombre de relations est proche de $B,$ et on
appelle $P$ la probabilité que $Y$ soit lisse. On a donc : 
\begin{alignat*}{2}
T & = && \frac{B}{P} \\
\log(T) & = && \log(B) - \log(P) \\
 & = && \log(B) - \log(u^{-u}) \\
 & = && \log(B) + u\log(u) \\
 & = && \log(B) + \frac{\log(C)}{\log(B)} \times \log(u) \\
 & = && \log(B) + \frac{\log(n)}{\log(B)} \times \log(u)
\end{alignat*}     

On étudie donc la fonction $f : x \mapsto x + \frac{\log(n)}{x},$ dont
la dérivée est 

\noindent $f' : x \mapsto 1 - \frac{\log(n)}{x^2}.$ $f$ atteint
son minimum en $x = \sqrt{\log(n)}.$ On choisit alors $\log(B) = \sqrt{\log(n)},$ donc $B = \text{e}^{\sqrt{\log(n)}}.$ 
\end{proof}

Avec un tel $B,$ on peut alors calculer $T.$ On a donc $u = \frac{\log(n)}{\sqrt{\log(n)}} = \sqrt{\log(n)}.$
\begin{alignat*}{2}
\log(T) & = && \log(B) + \frac{\log(n)}{\log(B)} \times \log(u) \\
 & = && \sqrt{\log(n)} + \sqrt{\log(n)} \times \log(\sqrt{\log(n)}) 
\end{alignat*}

$\log(\sqrt{\log(n)}) \text{ est négligeable devant } \sqrt{\log(n)},$
on a donc
\[\log(T) \approx (\sqrt{\log(n)})^{1+o(1)},\] puis
\[T \approx \exp(\sqrt{\log(n)}^{1+o(1)}).\]

Une analyse plus fine montre que $T = \exp(\sqrt{2\log(n)\log\log(n)}).$

\subsection{Limites de l'algorithme}

Contrairement à l'algorithme $p - 1$ de Pollard, la complexité dépend
ici de $n$ et plus du plus petit facteur premier de $n.$ Le crible de
Dixon est donc plus efficace pour factoriser des entiers de type $n =
pq$ avec $p$ et $q$ premiers (comme dans RSA, par exemple) que pour trouver des petits facteurs de
grands entiers.


\section{Crible quadratique}

L'algorithme du crible quadratique est un algorithme de factorisation fondé sur l'arithmétique modulaire, c'est-à-dire l'ensembles de méthodes, dérivées de l'étude du reste obtenu par une division euclidienne, permettant la résolution de problèmes sur les nombres entiers. En pratique c'est l'algorithme le plus rapide, sauf pour les nombres d'au moins cent chiffres décimaux, pour lesquels le crible algébrique est plus performant.
Le temps d'exécution du crible quadratique dépend uniquement de la taille de l'entier à factoriser, et non de propriétés particulières de celui-ci.

\subsection{Principe de l'algorithme}

L'algorithme, mis au point en $1981$ par Carl Pomerance, est un raffinement de la méthode de factorisation de Dixon. Le but est d'essayer d'établir une congruence de carrés modulo $n$ (l'entier à factoriser) qui nous permettra de factoriser $n.$

L'algorithme fonctionne en deux phases:
\begin{itemH}
\item la phase de collecte des données, où il collecte les informations qui peuvent conduire à une congruence de carrés, et
\item la phase d'exploitation des données, où il place toutes les données qu'il a collectées dans une matrice et la résout pour obtenir une congruence de carrés.
\end{itemH}

Pour la phase de collecte des données, on choisit un entier $m$ proche de la racine carrée de $n$ \[ m = \lfloor n^{\frac{1}{2}} \rceil .\]

Ensuite on forme des congruences modulo $n$ en observant que pour tout entier $a,$ \[ (m + a)^2 \equiv (m^2 - n) + a^2 + 2am \pmod n , \] où on note que $m^2 - n$ est de l'ordre de $\sqrt{n}.$

On se donne une borne $B$ et l'on cherche des petits entiers $a$ tels que $(m^2 - n) + a^2 + 2am \pmod n$ soit $B-$friable.

{\`A} partir de la décomposition en facteurs premiers des nombres trouvés $B-$friable on forme ce qu'on appelle des relations, qui sont de la forme \[(m^2 - n) + a^2 + 2am = b^r.c^s... \pmod n\] avec $b$, $c$ des premiers et $r$, $s$ des entiers.

Grâce à ces relations, on porte dans une matrice la parité des valuations et on forme des carrés à partir des lignes annulées par cette matrice. L'ensemble de ces lignes est un espace vectoriel.

On trouve donc une congruence de carrés comme $b^2 = c^2 \pmod n.$ Pour trouver un facteur $p$ de $n,$ on calcule $\pgcd(b - c, n).$

La durée d'exécution du crible quadratique pour factoriser un entier $n$ est en $O(\exp(\sqrt{\log n \log \log n})).$

\subsection{Pseudo-code}
Soit P l'ensemble des premiers inférieurs ou égaux à $B.$

\begin{algorithm}
\caption{Factorisation de $n$ par le crible quadratique}
\KwIn{Un entier $n$ et une borne $B$}
\KwOut{Un facteur de $n$}
\BlankLine
$m = \lfloor n^{\frac{1}{2}} \rceil$\;
$k = \#P$\;
$L = liste\_relations$\;
\For{$a \in \{-60,..., 60\}$}{$y = (m^2 - n) + a^2 + 2am \pmod n$\;
\If{$y \text{ }B \text{-friable }$}{$L[i] \gets  y$}}
$M = (v_{i, p} \pmod 2)_{a, p \in P} \text{ où } L[i] = \prod\limits_{p \in P}p^{v_{i,p}}$\;
$(e_1, ..., e_k) \text{ tel que } (e_1, ..., e_k)M = 0$\;
$u = \prod\limits_{p \in P} p^{\frac{1}{2} \sum\limits_{i=1}^k v_{i, p}e_i}$\;
$v = \prod\limits_{\substack{a | e_i \neq 0 \\ 1 \leq i \leq k}} (m + a)^2$\;
\Return{$\pgcd(u - v, n)$}
\end{algorithm}

\subsection{Exemple}

Soit \[n = 21311 = 101 \times 211\] le nombre à factoriser. On choisit l'entier $m$ proche de la racine carrée de $n$ \[ m = \lfloor n^{\frac{1}{2}} \rceil = 146 .\]
On forme les congruences modulo $n$ tels que \[  (m + a)^2 \equiv (m^2 - n) + a^2 + 2am \pmod n = 5 + a^2 + 292a \pmod{21311} .\]
On prend comme borne $B = 13$ puis on va donc chercher pour quels $a$ l'entier $5 + a^2 + 292a$ est $B-$friable. Prenons par exemple $a$ compris entre $-60$ et $60$.
On trouve:

\begin{center}
\begin{tabular}{c|c}
$a$ & $5 + 292a + a^2$\\
\hline
$-27$ & $-2.5^2.11.13$\\
$ -5$ & $-2.5.11.13$\\
$-1$ & $-2.11.13$\\
$0$ & $5$\\
$60$ & $5^3.13^2$\\
\end{tabular}
\end{center}

D'où la matrice représentant la parité des valuations:

\[ M = 
\begin{tabular}{c|ccccc}
 & $-1$ & $2$ & $5$ & $11$ & $13$\\
\hline
$-27$ & $1$ & $1$ & $0$ & $1$ & $1$\\
$ -5$ & $1$ & $1$ & $1$ & $1$ & $1$\\
$-1$ & $1$ & $1$ & $0$ & $1$ & $1$\\
$0$ & $0$ & $0$ & $1$ & $0$ & $0$\\
$60$ & $0$ & $0$ & $1$ & $0$ & $0$\\
\end{tabular}
\]

Grâce aux lignes annulées par cette matrice, qui forment un espace vectoriel dont une base est donnée par les trois lignes de la matrice suivante, on forme des carrés:

\begin{center}
\begin{tabular}{ccccc}
 $-27$ & $-5$ & $-1$ & $0$ & $60$\\
\hline
$1$ & $0$ & $1$ & $0$ & $0$\\
$1$ & $1$ & $0$ & $1$ & $0$\\
$1$ & $1$ & $0$ & $0$ & $1$\\
\end{tabular}
\end{center}

La première ligne du tableau donne la congruence \[ (2.5.11.13)^2 \equiv (146 - 27)^2.(146 - 1)^2 \pmod{21311} .\]

On calcule donc le pgcd de \[ 2.5.11.13 - (146 - 27).(146 - 1) = -15825 \] et de $21311.$ On trouve le facteur $p = 211$ de $n = 21311$ et son cofacteur $q = 101.$
Après un test de primalité, on prouve que $p$ et $q$ sont premiers.

\section{Conclusion}

Les algorithmes de factorisation d'entiers ont beaucoup évolué au fil du temps, notamment depuis l'apparition de la cryptologie.
Actuellement, l'algorithme le plus performant pour les très grands entiers (plus de cent chiffres décimaux) est le crible algébrique, appelé aussi crible général de corps des nombres,
qui est une amélioration du crible quadratique.
Sa complexité est en $O(\exp(\sqrt[3]{\frac{64}{9}\log n (\log \log n)^2})).$

Afin de motiver la recherche d'algorithmes de factorisation plus performants, RSA Laboratories a mis en place le 18 mars 1991 le ``RSA Factoring Challenge'', une compétition de factorisation d'entiers RSA.
Un entier RSA est un entier $n$ de la forme $n = pq$ avec $p$ et $q$ premiers.
RSA Laboratories a donc publié une liste d'entiers RSA de tailles allant de cent décimaux (trois cent trente bits) à six cent dix-sept (deux mille quarante-huit bits),
ainsi qu'une récompense financière pour la factorisation réussie de ces entiers.
Le plus grand entier RSA a avoir été factorisé (le 12 décembre 2009) faisait deux cent trente-deux décimaux (soit sept cent soixante-huit bits).
C'est pour cette raison que les clefs RSA sont de taille allant de mille vingt-quatre bits à deux mille quarante-huit bits, car des clefs de cette taille ne semblent pas près d'être factorisées.

Toutefois, le développement d'ordinateurs quantiques pourrait remettre tout ceci en question. En effet, avec de tels ordinateurs, il serait alors possible de factoriser des entiers RSA en temps polynomial,
rendant alors le chiffrement RSA quasiment inutile.

\end{document}
