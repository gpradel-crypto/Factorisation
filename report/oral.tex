 \documentclass[utf8,10pt,french]{beamer}
\usepackage[french]{babel}
\usepackage{listings}
\usepackage{amsfonts}
\usepackage{color}
\usepackage{graphicx}
\usepackage{amsmath}
\usepackage{amsthm}
\usepackage{amssymb}
%\usepackage{listings}
\usepackage{verbatim}
%\input{defs.tex}


\usetheme{Singapore}
%\usecolortheme{wolverine}
%\usetheme{Copenhagen}
%\usetheme{Hannover}

\newcommand{\nam}[1]{\texttt{\color{blue}#1}}
\newcommand{\com}[1]{\texttt{\color{red}#1}}
\newcommand{\var}[1]{\textit{\color{green!50!black}#1}}

\title[FdE]{Factorisation des entiers}
\author[GV]{\textsc{Gaëtan Pradel \& Vincent Dalsheimer}}
\institute{Université Bordeaux 1}


\newcommand{\R}{\mathbb{R}}
\newcommand{\Z}{\mathbb{Z}}
\newcommand{\N}{\mathbb{N}}

\DeclareMathOperator{\pgcd}{pgcd}

\begin{document}

\frame{\titlepage}

\section{Introduction}
\begin{frame}
  \frametitle{Introduction}
\begin{itemize}
\item Un ordinateur est déterministe : étant donné des paramètres d'entrée, il produira toujours le même résultat. \pause
\item Un générateur pseudo-aléatoire est un algorithme qui permet de générer une suite de nombres qui semble aléatoire à partir d'une suite plus courte. Aucune machine ne peut distinguer la suite produite d'une véritable suite aléatoire. \pause
\item On distingue deux catégorie d'exigences : 
\begin{itemize}
\item contexte de simulation simple : éviter les caractères moutonniers avec une loi de probabilité bien distribuée;
\item contexte cryptographique : non prévisible pour faire face à un adversaire;
\end{itemize}
\end{itemize}
\end{frame}

\section*{Sommaire}
\begin{frame}
  \frametitle{Sommaire}
  \tableofcontents
\end{frame}

\section{Methode naïve de génération d'aléa}
\subsection{Générateur Congruentiel Linéaire}
\begin{frame}
  \frametitle{Générateur Congruentiel Linéaire}
\begin{definition}
La suite $(X_{n})$ est définie de la façon suivante:
\begin{center}
$X_{n+1}=(a\times X_{n} + b) \mod{m} $ 
\end{center}
\end{definition}\pause
  
\begin{example}
Soit $a=238$, $b=12$,  $m=1009$ et $X_0=456$  : \\
\begin{itemize}
\item $X_1= a \times X_0 + b = 238 \times 456 + 12 = 577$
\item $X_2= a \times X_1 + b = 238\times 577 + 12$ = $114$
\item $X_3= 238 \times 114 + 12$ = $910$
\end{itemize}
Nous obtenons la suite suivante :
\begin{center}
(577,114,910,666,107,253,...)
\end{center}
\end{example}
\end{frame}

\subsection{Attaque sur les générateurs}
\begin{frame}
  \frametitle{Attaque sur les générateurs}
 En suppposant que \textit{$m$ soit connus} :
$$\begin{cases}
a=(x_{n}-x_{n+1})^{-1}\times(x_{n+1}-x_{n+2}) \mod{m}  \\
b=(x_{n}-x_{n+1})^{-1}\times(-x_{n+1}^2 + x_{n} \times x_{n+2}) \mod{m}
\end{cases}$$ \pause
Maintenant si \textit{m inconnue} : \\
On pose $\delta_{n}=(x_{n+2}-x_{n+1})(x_{n}-x_{n-1})-(x_{n+1}-x_{n})^2$.
$\delta_n$ est un multiple de $m$.\\ $m$ sera alors égale au $\pgcd$ des $\delta_{n},\delta_{n+1},\delta_{n+2},...$.\pause

\begin{example}
Soit la suite : 577,114,910,666,107,253,...
\begin{itemize}
\item $\delta_{n} = (666-910) \times (114-577) - (910-114)^2 = -520644$
\item $\delta_{n+1} = (107-666) \times (910-114) - (666-910)^2 = -504500$
\item $\delta_{n+2} = (253-107) \times (666-910) - (107-666)^2 = -348105$
\end{itemize}
Alors $\pgcd(\delta_n,\delta_{n+1})=4036$, et $\pgcd(4036,\delta_{n+2})=1009$
\end{example}
\end{frame}

\section{Notions élémentaires}
\subsection{Langage}
\begin{frame}{Langage}
Un langage est un ensemble de mots sur un alphabet fini.\\
Un alphabet est un ensemble de symboles et un mots est une suite finie de ces symboles.\\
Un langage ne modélise pas un calcul mais un problème. En effet, si l'on introduit un langage comme un sous ensemble de $\{0,1\}^*$, un problème associé à celui-ci sera de dire si un mot appartient à ce langage ou non.\\
Ceci est un \var{problème décisionnel}.
\end{frame}

\subsection{Deux classes de problèmes}

\begin{frame}{Deux classes de problèmes}
En informatique, il existe deux grandes classes de problèmes :
\begin{itemize}
\item les problèmes décisionnels,
\item les problèmes fonctionnels.
\end{itemize}
Les problèmes fonctionnels consistent à calculer l'image d'un mot par une application de $\{0,1\}^*$ dans $\{0,1\}^*$.\\
Pour résoudre les problèmes décisionnels, nous introduisons la notion d'\nam{automate}.
\end{frame}

\subsection{Automates}
\begin{frame}{Automates}
Un automate est un système qui permet de résoudre des problèmes décisionnels.\\
\begin{definition} 
Un automate fini déterministe $\mathcal{A}$  sur un alphabet A est un quadruplet $\mathcal{A}=(Q,F,I,T)$, où:
\begin{itemize}
	\item $Q$ est un ensemble fini d'états,
	\item $F \subset Q \times A \times Q$ est la \var{règle de transition},
	\item $I \in Q$ est l'\var{état initial},
	\item et $T \subset Q$ est un ensemble d'\var{états finaux}, terminaux ou acceptants.
\end{itemize}
\end{definition}
\end{frame}

\begin{frame}{Automates}
L'automate se place dans l'état initial, passe d'un état à un autre, en lisant les mots lettre par lettre, selon la règle de transition. Lorsque le mot est lu, celui-ci est accepté dans l'automate seulement si le dernier état dans lequel il se trouve est un état final. \\
\begin{center}
\end{center}
Par exemple, le mot $bbbabbbaaaabaabb$ est accepté par cet automate, tandis que $abababbab$ ne l'est pas.\\
Ces automates présentent néanmoins une faille: en effet, \com{FAUT QUE JE METTE L'EXEMPLE DU PROF AVEC LE FICHIER AUDIO}. Pour faire face à ce problème, Nous avons ce qu'on appelle les machines de Turing, qui sont des automates sous la forme d'une bande infinie à droite et qui possède de la mémoire. Grâce à elles, nous avons les \var{algorithmes}. 
\end{frame}

\subsection{Complexité}
\begin{frame}{Complexité}
Les algorithmes sont des procédures automatiques qui résolvent des problèmes. La \var{complexité} est une caractérisation de la qualité d'un algorithme : c'est l'ordre de grandeur du nombre d'opérations binaires effectuées par la machine de Turing en fontion de la taille des entrées.\\
Voici quelques complexités : 
\begin{itemize}
\item Complexité linéaire: \com{$O(n)$},
\item Complexité quadratique: \com{$O(n^2)$},
\item Complexité polynomiale: \com{$O(n^k)$},
\item Complexité sous-exponentielle si $0 \leq k \leq 1$ et exponentielle si $k \ge 1$: \com{$O(e^{n^k})$}. 
\end{itemize}
On considère qu'un algorithme est efficace lorsqu'il opère avec une complexité au plus polynomiale.
\end{frame}

\begin{frame}{Complexité}
Voici un exemple de calcul de complexité : le produit matriciel de deux matrices carrés de taille $n$.
\begin{itemize}
\item Il y a $n \times n = n^2$ coefficients à calculer.\pause
\item Pour chaque coefficient, il y a $n$ produits et $n-1$ sommes à faire.\pause
\item Il y a donc $n^2 \times (n + n - 1) = 2n^3 - 1$ opérations à effectuer.\pause
\item On dit donc que la complexité est en $O(n^3)$.
\end{itemize}
On considère donc le calcul matriciel efficace.
\end{frame}

\section{Existence de générateurs pseudo-aléatoires}
\subsection{Fonctions à sens unique}
\begin{frame}
 \frametitle{Fonctions à sens unique}
\begin{definition}
Une fonction f: $\{0,1\}^* \longmapsto \{0,1\}^*$ est appelée à sens-unique si les conditions suivantes sont vérifiées:
\begin{itemize}
\item Facile à calculer: il existe un algorithme polynomial A tel que $A(x) = f(x)$ pour tout $x$ appartenant à $\{0,1\}^*$.
\item Difficile à inverser : il n'existe pas d'algorithme efficace qui calcule la fonction réciproque de $f$.
\end{itemize}
\end{definition} \pause
\begin{example}
Un candidat est celui de la factorisation d'un entier. Soient deux nombres premiers $p$ et $q$. Calculer $x=pq$ est facile même si 
$p$ et $q$ sont très grands. Par contre, les meilleurs algorithmes prouvés aujourd'hui pour retrouver $p$ et $q$ à partir de $x$ a un temps de calcul qui cro{\^i}t en $O(e^{\sqrt{b}})$ ou $O(exp(b^{1/2 + o(1)}))$ où $b$ est le nombre de bits de $x$.
\end{example}
\end{frame}

\begin{frame}
\begin{definition}
Soit $f: \{0,1\}^* \longmapsto \{0,1\}^*$ une fonction à sens unique. Une fonction $b: \{0,1\}^* \longmapsto \{0,1\}$ est un bit difficile pour $f$ si:
\begin{itemize}
\item $b(x)$ est calculable en un temps polynomial,
\item Il n'existe pas d'algorithme probabiliste polynomial efficace $A'$ qui prédit $b(x)$ étant donné $f(x)$. On ne peut pas retrouver d'informations sur un antécédant.
\end{itemize}
\end{definition} 
\end{frame}

\begin{frame}
\begin{example}
Prenons comme fonction à sens unique : $f:x \mapsto x^2 \mod{N}$, avec $N=pq$ où $p$ et $q$ sont des premiers distincts plus grands que $2$. Dans ce cas, un bit difficile peut être le bit de parité de $x$. \pause
 Le théorème chinois nous dit que l'application qui à $x$ associe le couple ($x \pmod{p}$, $x \pmod{q}$) est un isomorphisme d'anneaux: $$ \Z /N \Z \simeq \Z/p \Z \times \Z /q \Z$$ 
Si on connaît $f(x)$, on peut retrouver $x \pmod{p}$ et $x \pmod{q}$. Ce qui induit 4 systèmes possibles pour trouver une solution modulo $N$ :
$\begin{cases}
x=x_p \mod{p}\\
x=x_q \mod{q} 
\end{cases}$ \pause
$\begin{cases}
x=-x_p \mod{p}\\
x=x_q \mod{q}
\end{cases}$ \pause
$\begin{cases}
x=x_p \mod{p}\\
x=-x_q \mod{q}
\end{cases}$ \pause
$\begin{cases}
x=-x_p \mod{p}\\
x=-x_q \mod{q}
\end{cases}$ \pause
 donc 4 solutions distinctes modulo $N$ : 
\begin{center}
$x$, $-x$, $y$, $-y$
\end{center}
\end{example}
\end{frame}

\begin{frame}
\begin{example}
Deux de ces solutions sont paires et les deux autres sont impaires :
Comme on raisonne modulo $N$, $-x=N-x$. Par construction, $N$ est impair.
\begin{itemize}
\item Si $X$ est impair, alors $N-X$ est de la forme $2k+1-2k'-1=2(k-k')$. Ainsi, $N-X$ est clairement pair. \pause
\item Si $X$ est pair, alors $N-X$ est de la forme $2k+1 - 2k'=2(k-k')+1$. Ainsi, $N-X$ est clairement impair. \pause
\end{itemize}
On en conclut que $x^2$ peut avoir été obtenu avec un $x$ pair mais aussi avec un $x$ impair. Ainsi, connaître $x^2$ ne nous informe pas sur la parité de $x$.
\end{example}
\end{frame}

\subsection{Construction d'un générateur pseudo-aléatoire}

\begin{frame}
  \frametitle{Construction d'un générateur pseudo-aléatoire}
\begin{definition}
Soit $b$ un bit difficile d'une fonction bijective $f$. Alors, $G(s) = f(s) \cdot b(s)$ est un générateur pseudo-aléatoire.
\end{definition} \pause
\begin{example}
Avec l'exemple ci-dessus : $f:x \mapsto x^2 \mod{N}$ et $b(x)$ le bit de parité de $x$, on obtient après une itération : $G(x)=x^2 \cdot b(x)$. On a vu que $x^2$ possède quatres antécédants dont deux pairs et deux impairs. Ainsi, la connaissance de $x^2$ ne nous dit rien sur $b(x)$, on ne peut donc pas anticiper le prochain bit qui sera ajouté à la prochaine itération. 
\end{example}
\end{frame}

\begin{frame}
\begin{example}
Voici ce que nous donne ce générateur après trois itérations :
 \begin{align}
1^{\mbox{ère}} \mbox{itération}: &G(s)=(s^2 \mod{N}) . b(s)\\
2^{\mbox{ème}} \mbox{itération}: & G(G(s))= (s^4 \mod{N}) . b(s^2) . b(s) \\
3^{\mbox{ème}} \mbox{itération}: & G(G(G(s)))=(s^8 \mod{N}). b(s^4) . b(s^2) . b(s)\\
...
\end{align}
\end{example}
\end{frame}

\subsection{Le générateur de Blum Blum Shub}

\begin{frame}
\frametitle{Le générateur de Blum Blum Shub}
\begin{definition}
On appelle premier de Blum, tout nombre premier congru à $3 \pmod{4}$.
\end{definition} \pause
\begin{definition}
On appelle entier de Blum, tout nombre $N=pq$ tel que $p$ et $q$ soient deux premiers de Blum.
\end{definition} \pause
\begin{example}
$21=3 \times 7$. En effet, $7=3 \pmod{4}$ ainsi $3$ et $7$ sont deux premiers de Blum et leur produit $21$ forme un entier de Blum.
\end{example} 
\end{frame}

\begin{frame}
\begin{definition}
Soit $N$ un entier de Blum ayant $k$ bits. On définit deux suites à partir d'un germe $X_{0}$: la suite des uniques $X_{i}=X_{i-2}^{2} \pmod{N}$ compris entre $0$ et $N-1$, et enfin celle des uniques $b_{i}=X_{i} \pmod{2}$. La suite des $b_{i}$ correspond à la suite des éléments engendrés par le générateur BBS (de Blum Blum Shub).
\end{definition} \pause
\begin{example}
Prenons $N=21$ et $X_0=3$ donc $b_0=1$, alors 
\begin{itemize}
\item $X_1=3^2=9 \mod{21} \mbox{ et } b_1=1 $\\
\item $X_2=9^2=81=18 \mod{21} \mbox{ et } b_2=0 $\\
\item $X_3=18^2=324=9 \mod{21} \mbox{ et } b_3=1$
\end{itemize}
Ainsi, on obtient la suite suivante : $101$.
\end{example}
\end{frame}

\begin{frame}
Nous avons vu que la suite obtenu avec la fonction à sens unique $f:x \mapsto x^2 \mod{N}$ et $b(x)$ le bit de parité de $x$ après trois itérations ressemble à ceci :
 \begin{align}
1^{\mbox{ère}} \mbox{itération}: &G(x)=(x^2 \mod{N}) . b(x)\\
2^{\mbox{ème}} \mbox{itération}: & G(G(x))= (x^4 \mod{N}) . b(x^2) . b(x) \\
3^{\mbox{ème}} \mbox{itération}: & G(G(G(x)))=(x^8 \mod{N}). b(x^4) . b(x^2) . b(x)
\end{align} \pause
Remarquons que la suite engendrée par le générateur de Blum Blum Shub correspond aux suites que nous venons d'obtenir à la différence près qu'elle est inversée et qu'elle ne garde pas $x^8 \mod{N}$. Après trois itérations, la suite générée par BBS ressemble à ceci : 
\begin{center}
$b(x).b(x^2).b(x^4)$
\end{center} \pause
Le générateur BBS est donc cryptographiquement sûr mais bien plus lent à calculer que le générateur cronguentiel linéaire.
\end{frame}




\section{Conclusion}
\begin{frame}
  \frametitle{Conclusion}
Nous avons deux façon de produire une suite de nombres aléatoire :\pause
\begin{itemize}
\item Générateur congruentiel linéaire : rapide à l'implémentation mais facile à attaquer.\pause
\item Blum Blum Shub : cryptographiquement sûr mais bien plus lent que le GCL.
\end{itemize}
\end{frame}
\end{document}
